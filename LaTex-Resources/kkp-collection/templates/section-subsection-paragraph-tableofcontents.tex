%SECTION, SUBSECTION, PARAGRAPH, TABLE OF CONTENTS, AND SOME FORMATTING EXAMPLES

\documentclass[14pt, a4paper]{article} %14 pt indicates the font size of the prepared document
\usepackage[utf8]{inputenc} %indicates the encoding of the document
\usepackage{color} %this package enables the use of colors.

\title{Template - Section SubSection Paragraph Table-of-Contents}
\author{Kawshik Kumar Paul}
\date{\today}
% \date{21 March 2021}

\begin{document}
\maketitle
\tableofcontents %this command creates the table of contents with all numbered sections, subsections, etc.
\pagebreak %This will force the rest of the document to start in another page.

\section{Introduction}
This is the introduction section.

\subsection{Subsection 1.1}
This is subsection 1.1.

This sentence will begin in a new paragraph due to the blank line before it. This is a filler line.\par
However, this sentence begins in a new paragraph due to the \\par command used after the end of the last sentence. This is another filler line.

\subsubsection{Subsubsection 1.1.1}
This is subsubsection 1.1.1.

\subsection*{Unnumbered subsection}
This is an unnumbered subsection under Introduction. \textbf{This will NOT show up in the Table of Contents.} 

\emph{This sentence is an example of emphasis. Even though it is italic in this example, using another package might change it to something else, like underlined text.  \emph{Also, emphasis can be nested.} The previous sentence was an example of nested emphasis.} 

\textit{This sentence will always be italic.}
%\color{red} This is a red sentence. %This command will make the rest of document red (including this sentence) 
{\color{red} This is a red sentence.} %This command will only make this sentence red


\section*{Unnumbered Section Example}
This section should not be numbered. Using * after the section specifier prevents this numbering. This will NOT show up in the Table of Contents.
\LARGE{This sentence is LARGE.}
\Huge{This sentence is HUGE.}

\end{document}
